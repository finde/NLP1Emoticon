
Once the data is collected and preprocessed, its features are extracted. Selecting good features can significantly improve the performance in many cases, but this task is not a trivial one. Due to time constraints, certain basic features were selected to conduct this research. However, it is believed that future research can explore a bigger (or different) set of features. The selected features are based on previous work in this area \cite{fairytales}. Table ~\ref{featuretable} presents the features which were extracted and used for classification. Since most of the features' names are self-explanatory, only some of the features are discussed in details in the following paragraphs.


\begin{table}[h!]
\caption{Extracted features}

\begin{center}
\begin{tabular}{ll}
\multicolumn{1}{l}{\bf FEATURE NAME}  &\multicolumn{1}{l}{\bf DESCRIPTION}
\\ \hline \\
words & total number of words in the text \\
positive words & total number of positive words in the text \\
negative words & total number of negative words in the text \\
positive words hashtags & total number of positive words in the hashtags \\
negative words hashtags & total number of negative words in the hashtags \\
uppercase words & number of words containing only uppercase letters \\
special punctuation & number of special punctuation marks such as `!' and `?' \\
adjectives & number of adjectives in the text 
\end{tabular}{}
\label{featuretable}
\end{center}
\end{table}



\noindent Extracting features such as \textbf{positive words} and \textbf{negative words} requires the usage of a predefined lists of `positive' and `negative' words, which were downloaded from \cite{hu2004mining}. Both these lists contain words that are positive or negative and are used to help predicting the emoticon of a short message. While implementing this aspect, the question of what effect extremely positive and extremely negative words have arose. In order to incorporate this idea, a different score might be used for the different subcategories of positive or negative words. Due to time constraints, this could not be researched. However, this is interesting to research in the future.
\\
\\
Features such as the number of \textbf{adjectives} were extracted by first performing part-of-speech (PoS) tagging on each sentence and then counting the adjectives. Due to spelling and grammar errors in the messages, PoS tagging does not always perform  well. As a consequence, slight inaccuracies are possible for this feature, though not significant enough to negatively influence the performance. % Ask Hoang how to deal with this, as testing is difficult

