This report discusses emoticon prediction, based on a sequence of messages. In order to find messages that are related to one another and can transition in positivity to negativity, an entire conversation is observed. Each message is classified by not only considering the features of a single message, but also by considering the previous messages of the same author. This problem is a complex and interesting one, since it also takes into account the transitions from one state to another.

\noindent In order to find messages with such transitions, a corpora containing longer sequences of text messages are used - these might be conversations, such as an e-mail exchange or simply chat data, as long as the corpora contains a sufficient amount of emoticons. As emoticons appear in chat data more often, Ubuntu chat data \cite{ubuntudata} is used. Unfortunately, due to the lack of emoticons in this specific corpus, the emoticons of all sentences, which do not contain an emoticon, must also be determined. In order to relabel these sentences, another prediction model must be trained. In order to train this model, a corpus of short, unrelated Twitter \cite{twittersentiment} messages, containing emoticons, is used. This model is then used to relabel the Ubuntu chat data. 

\noindent After the Ubuntu chat data is collected, it is first preprocessed. Then, appropriate features are extracted and a model is trained. The current report does not discuss the choice of features in detail. Basic features were chosen for this project without any intentional aim on improving this selection. Since this task on its own can take years, it is left as future work. However, suggestions on how to improve these features are mentioned in the following sections.

\noindent Three basic classes are used in order to classify the data. Table \ref{tab:classes} presents these classes and the corresponding emoticons, contained in this type of data.

% Maybe put this in the general part, since it is the same for both problems?
\begin{table}[h!]
\caption{Data classes}
\label{tab:classes}
\begin{center}
\begin{tabular}{ll}
\multicolumn{1}{l}{\bf DATA CLASS}  &\multicolumn{1}{l}{\bf EXAMPLE EMOTICONS}
\\ \hline \\
positive 			& :) \: \:  			:] \: \: 		=) \: \: 		:D  	\\
neutral 				& :$\vert$ \: \: 	o\_o  								\\
negative 			& :( \: \: 			:[ \: \: 		=( \: \: 		;( 		\\
\end{tabular}{}
\end{center}
\end{table}

As table \ref{tab:classes} shows, this selection of emoticons displays positive, negative and neutral emotions. However, this data can be split further into extremely positive ( :D ) and extremely negative ( $;($ ) emoticons for more specific classification of messages. However, this is beyond the scope of this paper.

\begin{comment}
\red{Finish the introduction, maybe move some parts to the Problem section} \\



% Requirements: 
\begin{comment}
Introduction (max 2 pages):
• Description of the problem area and the problem itself
• What is the research question / goal?
• Why is this an important / meaningful / interesting problem to consider?
• The very basic idea of the approach and why this is a reasonable approach for this problem?
\end{comment}

\begin{comment}
This report discusses two related, but essentially different problems - emoticon prediction based on a single short message and based on a sequence of messages. In the first problem, short, unrelated messages are observed. Their emoticons are used to determine whether a message is positive or negative. Using this information, a is trained. Eventually, a new message can be passed and the corresponding emotion is predicted. In order to determine the sentiment of a message optimally and achieve generalization, messages of different authors are collected and used to train the classifier. In the second problem, an entire conversation is observed. Each message is classified by not only considering the features of a single message, but also by considering the previous messages of the same author. This problem is a more complex and interesting one, since it also takes into account the transitions from one sentiment state to another. \\

\noindent As a natural consequence of the above statements of the two problems, different data sets are used to approach each problem. For the first problem, any corpora of text messages containing emoticons can be used. Such corpora can contain unrelated messages. In order to find short unrelated messages, Twitter data is used. For the second problem, a corpora containing longer sequences of text messages are used - these might be conversations, such as an e-mail exchange or simply chat data, as long as the corpora contains a sufficient amount of emoticons. As emoticons appear in chat data more often, Ubuntu chat data is used. \\

\noindent After the data is collected, it is first preprocessed. Then, appropriate features are extracted and a model is trained. The current report does not discuss the choice of features in detail. Basic feature were chosen for this project without intentional aim on improving them. Since this task on its own can take up years, it is left as future work. However, suggestions about improving features is mentioned in the following sections. \\

\noindent In order to classify the data, classes must be determined. Three classes are used to classify the data. Table \ref{tab:classes} displays these classes and the corresponding emoticons, contained in this type of data.
\end{comment}

% Old and ugly
\begin{comment}
\begin{itemize}
% TODO: Fix the emoticon examples to match the emoticons actually used
\item \textbf{positive} - data containing `positive' emoticons (expressing positive emotions) such as `:)', `:]', `=)', or `:D'
\item \textbf{neutral} - data containing no emoticons or neutral emoticons like `:$\vert$' or `o\_o'
\item \textbf{negative} - data containing `negative' emoticons (expressing negative emotions) such as `:(', `:[', `=(` or `;(' 
\end{itemize}
\end{comment}

\begin{comment}
% Maybe put this in the general part, since it is the same for both problems?
\begin{table}[h!]
\caption{Data classes}
\label{tab:classes}
\begin{center}
\begin{tabular}{ll}
\multicolumn{1}{l}{\bf DATA CLASS}  &\multicolumn{1}{l}{\bf EXAMPLE EMOTICONS}
\\ \hline \\
positive 			& :) \: \:  			:] \: \: 		=) \: \: 		:D  	\\
neutral 				& :$\vert$ \: \: 	o\_o  								\\
negative 			& :( \: \: 			:[ \: \: 		=( \: \: 		;( 		\\
\end{tabular}{}
\end{center}
\end{table}
As table \ref{tab:classes} shows, known emoticons display positive, negative and neutral emotions. However, this data can be split further into extremely positive ( :D ) and extremely negative ( >:( ) emoticons for more specific classification of messages. However, this is beyond the scope of this paper and is further mentioned in future work.
\ldots \ldots \ldots 

\red{Finish the introduction, maybe move some parts to the Problem section} \\
%%% Continue the rest of the discussion of the problem by only mentioning something brief about the models and possible alternatives. More details are in the corresponding sections!!
\end{comment}
