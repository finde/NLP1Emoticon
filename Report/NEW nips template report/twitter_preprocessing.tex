
Before any of the data can be used, the messages need to be preprocessed. This process includes three basic steps. Each one of them is discussed in more details in the following paragraphs.

% I saw that this is how people put such not really `important' subsubsections instead of bullets
\subsubsection*{Splitting the message.}
The message, itself, is split into different components that will help in training. In this case, the text, hashtags and emoticons are extracted. Emoticons in messages are used to add emotion, but not text. The text itself can be used to determine whether a message is positive, negative or neutral. Therefore, the text without the emoticons can be used to determine which emoticon can be assigned to a certain message. Also, the hashtags can be used to further determine the emotion of the message. Hashtags often contain extra information, such as 'sad' or 'ecstatic'. This can also help determine the emotion of a message. In this case, however, the hashtags were not used. Therefore, it is important to research the effect of this feature in the future. 

\subsubsection*{Spelling and grammar correction}
Short messages are often posted as a reaction or something to share with friends quickly. This means that many messages contain spelling and grammar errors. In order to optimally use the Twitter data, all messages must be fixed spelling and grammar wise. There are several ways of performing these corrections, however these fall beyond the scope of this research. Therefore, names, links, emoticons and hashtags are disregarded when processing data, but the spelling and grammar errors remain. It is imperative to solve these errors in the future for optimal performance.

%\subsubsection*{Grammar check and correction}

