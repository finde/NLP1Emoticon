In this part of the project the Ububntu Chat Corpus was used as training data \red{SOME REFERENCE HERE}. This corpus consists of archived chat logs  from Ubuntu's Internet Relay Chat technical support channels \red {Reference}. The logs contain continuous conversations, where numerous people take part in the conversation. Some people only post a couple of messages, whereas others take more active part in the conversations.  \\ \\

\subsubsection*{Relabeling the data.} 
Since this data contains mainly technical conversations, there appeared to be a significant lack of emoticons. This means that the majority of the messages were labeled as `neutral', which in turn implies a horrible prediction performance. In oder to still be able to use this data and test the prediction performance, a decision was made to relabel the data. This was done in the following manner: for every `neutral' message an emoticon was predicted using the classifier discussed in section \ref{sec:twitter}. This lead to the possession of data containing an amount of emoticons enough to properly train a model and predict new examples. In order to avoid such artificial relabeling of the data in other projects, corpora from a more appropriate source should be used - e.g. Skype conversation logs or any other group chat logs.  
