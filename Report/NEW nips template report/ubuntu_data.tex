In this part of the project the Ubuntu Chat Corpus was used as training data \cite{ubuntudata}. This corpus consists of archived chat logs from Ubuntu's Internet Relay Chat technical support channels, as referred to in \cite{ubuntudata}. The logs contain continuous conversations, where numerous authors take part in the conversation. Some authors only post few messages, whereas others participate more actively in the conversations. 

\subsubsection*{Relabeling the data} 
Since this data contains mainly technical conversations, there appeared to be a significant lack of emoticons. This means that the majority of the messages were labelled as `neutral', which in turn implies a terribly biased prediction performance. In order to use this data and test the prediction performance, the decision was made to relabel the data. For every `neutral' message, an emoticon was predicted using the AMP discussed in section \ref{sec:twitter}. This lead to the acquisition of data containing enough emoticons to properly train a model and predict new examples. In order to avoid such artificial relabelling of the data in other projects, corpora from a more appropriate source should be used - e.g. Skype conversation logs or any other group chat logs.
