% \red{To be honest the next 3 paragraphs are also a bit confusing}
There are two models that are implemented to classify single message into a certain class: multiclass perceptron and multilayer perceptron.
Perceptron is an algorithm for supervised classification that classify an input to one of binary classes which only work well linearly separable data.

A multiclass perceptron is a generalized perceptron algorithm that allows supervised classification to be classified as one of many classes.
Basically, a multiclass perceptron finds a border between a class against the other classes.
For every iteration, there will be a new border drawn for each class.
Even if it similar, the new border is not the same as the previous iteration.
One way to improve the performance of multiclass perceptron is to average these border per class, instead of take the last border only.
Since the Twitter data has to be classified as positive, neutral and negative before assigning an emoticon to the message, this algorithm does as desired.

However, as explained before, perceptron algorithm are not able to solve data which are not linearly separable.
Hence, hidden layers are added to the perceptron.
These hidden layers help classify the data more accurately.
Unfortunately, this model is very sensitive to change.
Therefore, tuning the model is very difficult and lead to an unstable model.
In order to have a more stable model, the AMP is used to label the sequential messages.
