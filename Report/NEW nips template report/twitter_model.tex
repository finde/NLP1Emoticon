
A message can be classified as a positive, negative or neutral emotion. In order to classify these messages, an averaged multi-class perceptron was implemented. A multiclass perceptron is an algorithm that allows supervised classification. As the name of the algorithm implies, the input does not have to be binary allowing input to be classified as one of many classes. As the Twitter data has to be classified as positive, neutral and negative before assigning an emoticon to the message, this algorithm does as desired. 

Basically, a multiclass perceptron finds a border between two classes and draws a border between the two classes. By repeating this, several borders will appear between classes and are used to classify input. This leads to certain "gray"-areas, as different borders assign input data to different classes. In the case of the multiclass perceptron, the average border of all is computed. This creates one border, without any "gray"-areas.
