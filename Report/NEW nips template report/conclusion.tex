The selected corpora for this research was not sufficient and needed labelling in order to find the positivity, negativity or neutrality of a message. This means that the selected corpora was not sufficient for the research at hand and must be selected better in the future.

The preprocessing of the corpora was effective, but can be improved in the future. Minimal correction grammar- and spelling errors was sufficient enough to find positivity, negativity or neutrality in a message. Perhaps when extreme positivity and negativity are also taken into account, these corrections need to be more specific. For the this research, this was enough. 

All tests have shown that the selected features help predict emoticons. However, when excluding features, no exclusion lead to a dramatic rise or drop in accuracy. This means that the selected features are not sufficient and other and additional features must be determined for better performance. 

When selecting the amount of clusters to determine the transition and emission probabilities, an increasing amount of clusters leads to worse results. It appears as though the model overfits. However, as the best performance and the worst performance with a certain amount of clusters hardly seems to make a difference, it is believed that there is a bug in the the implementation. Due to time constrains, it isn't possible to find this bug and to solve this problem.

% Requirements:
\begin{comment}
Discussion and Conclusions (0.5 – 1 page)
• Refer to the research questions you defined in your introduction.
• Any related work you are aware of?
• Challenges you observed?
• “Future work” (you do not need to do this work really J, but what would you change in the
model / what experiments you would run / etc, if you would have a chance to do this? What other
people should look into?
• Any thoughts / observation / wider implications
\end{comment}
