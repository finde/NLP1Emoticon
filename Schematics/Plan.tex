\documentclass{article}

\usepackage[margin=1in]{geometry}
\usepackage{amsmath,amssymb}
\usepackage{verbatim}
\usepackage{graphicx}

%\usepackage{soul}

\begin{document}

\title{Predicting emoticons in text \\ Weekly plan}

\author{By Gornishka, Loor, Qodari, Xumara}
\maketitle

\section*{Week 45 (current week)}

\subsubsection*{Priority:} 
Get a clear idea about the project: 

\begin{itemize}
\item[1.] how much data do we need - start from around 1000-2000 training tweets and increasing the number in case it appears both possible and needed.   \\
\item[2.] how should we pre-process the data - remove hyperlinks and tags; detect the language and make sure it's English; perform spelling and grammar correction; store hashtags separately - they might be useful as features (e.g. tweet containing the hashtag \#BestDayEver is more likely to also contain ":)" instead of ":(" ) \\
\item[3.] decide on "emoticon classes" - start with \textbf{Positive} ( :) , ;) , :D, :] ), \textbf{Negative} ( :( , ;( , :[ ) and \textbf{Neutral} ( :| , :/ ) and increase the number of classes once the project has progressed enough (possibly extremely positive (:D), extremely negative ( ;) ), surprised ( :O ) or other, depending on the temporal results). \\
\item[4.] which algorithms are we going to use for classification - possibly \textbf{CRF}, \textbf{logistic regression} or \textbf{SVM}. Implement at least one and use at least one more in order to compare results. \\
\item[5.] what features might be useful - word count; count of positive and negative words in the tweet; count of positive and negative words in the hashtags; special punctuation used (?, !); upper-case words; count of adjectives or other parts of the speech; Expand the list as much as possible through the whole project.  
\end{itemize}

In order to be sure about the points above, research previous work related to the project during week 45 and week 46.


\section*{Week 46:} 
\subsubsection*{Priorities:}
\begin{itemize}
\item[1.] Finalize the project idea. 
\item[2.] Effort on collecting and pre-processing the data. 
\item[3.] Implement basic functions for feature-extraction.
\end{itemize}


\section*{Week 47:} 
\subsubsection*{Priorities:}
\begin{itemize}
\item[1.] Implement a classifier and possibly get a first working version of the project. 
\item[2.] Discuss the progress so far. Decide if more data is needed, if the chosen classifiers yield a good performance and what features seem to be important so far.
\end{itemize}


\section*{Week 48:} 
\subsubsection*{Priorities:}
\begin{itemize}
\item[1.] Explore more features and adjust parameters in order to improve the performance. 
\end{itemize}


\section*{Week 49:} 
\subsubsection*{Priorities:}
\begin{itemize}
\item[1.] Prepare a demo. 
\item[2.] Fix possible bugs. 
\item[3.] Collect results for the presentation and the final report.
\end{itemize}


\section*{Week 50:} 
\subsubsection*{Priorities:}
\begin{itemize}
\item[1.]  Prepare a presentation.
\end{itemize}


\section*{Week 51:} 
\subsubsection*{Priorities:}
\begin{itemize}
\item[1.] Finalize the project.
\item[2.] Prepare final report. 
\end{itemize}


\begin{comment}
Sample table in case it's needed
\begin{center}
	\begin{tabular}{ l | l | l | l | l | l || l | l | l | l || l}
		\textbf{K} & Sampling & C-space & VocabSize & TrainSize & Kernel & Airplanes & Cars & Faces & Motorbikes & MAP \\ 
		\hline
		400 & dense & gray & 100 & 300 & linear & 87 & 74 & 95 & 69 & 81 \\
		800 & dense & gray & 100 & 300 & linear & 93 & 70 & 86 & 73 & 81 \\
		1600 & dense & gray & 100 & 300 & linear & 88 & 72 & 86 & 64 & 78 \\
		2000 & dense & gray & 100 & 300 & linear & 86 & 73 & 81 & 51 & 73 \\	
		4000 & dense & gray & 100 & 300 & linear & 90 & 75 & 83 & 53 & 75 \\	
	\end{tabular}
\end{center}

\end{comment}


\begin{comment}
\begin{itemize}
	\item MatLab (http://www.mathworks.nl/products/matlab/)
	\item VL\_Feat (http://www.vlfeat.org/)
	\item LibSVM (http://www.csie.ntu.edu.tw/cjlin/libsvm/)
	\item Caltech Vision Group (http://www.vision.caltech.edu/)
\end{itemize}
\end{comment}

\end{document}
